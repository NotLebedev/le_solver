\documentclass[a4paper,12pt,titlepage,finall]{article}

\usepackage[T1,T2A]{fontenc}     % форматы шрифтов
\usepackage[utf8]{inputenc}      % кодировка символов, используемая в данном файле
\usepackage[english, russian]{babel}      % пакет русификации
\usepackage{tikz}                % для создания иллюстраций
\usepackage{pgfplots}            % для вывода графиков функций
\usepackage{geometry}		     % для настройки размера полей
\usepackage{indentfirst}         % для отступа в первом абзаце секции
\usepackage{amsmath,amsthm,amssymb}
\usepackage{mathtext}
\usepackage{graphicx}
\usepackage{hyperref}
\graphicspath{ {./} }

%Настройка листингов для языка C
\usepackage{xcolor}
\usepackage{listings}
\lstset{extendedchars=\true}

\definecolor{mGreen}{rgb}{0,0.6,0}
\definecolor{mGray}{rgb}{0.5,0.5,0.5}
\definecolor{mPurple}{rgb}{0.58,0,0.82}
\definecolor{backgroundColour}{rgb}{0.95,0.95,0.95}

\lstdefinestyle{CStyle}{
    backgroundcolor=\color{backgroundColour},   
    keywordstyle=\color{mGreen},
    numberstyle=\tiny\color{mGray},
    breakatwhitespace=false,         
    breaklines=true,                 
    captionpos=b,                    
    keepspaces=true,                 
    numbers=none,                    
    numbersep=5pt,                  
    showspaces=false,                
    showstringspaces=false,
    showtabs=false,                  
    tabsize=2,
    language=C,
    basicstyle=\footnotesize\ttfamily ,
    extendedchars=\true ,
}

% выбираем размер листа А4, все поля ставим по 3см
\geometry{a4paper,left=30mm,top=30mm,bottom=30mm,right=30mm}

\setcounter{secnumdepth}{0}      % отключаем нумерацию секций
\setcounter{tocdepth}{2}

\usepgfplotslibrary{fillbetween} % для изображения областей на графиках

\begin{document}
\begin{titlepage}
    \begin{center}
	{\small \sc Московский государственный университет \\имени М.~В.~Ломоносова\\
	Факультет вычислительной математики и кибернетики\\}
	\hrulefill
	\vfill
	{\large \bf Компьютерный практикум по учебному курсу}\\
	~\\
	{\Large \bf <<ВВЕДЕНИЕ В ЧИСЛЕННЫЕ МЕТОДЫ>>}\\ 
	~\\
	~\\
	~\\
	{\Large \bf ЗАДАНИЕ № 1}\\
	~\\
	{\large \bf ОТЧЕТ}\\
	{\bf о выполненном задании}\\
	{студента 201 учебной группы факультета ВМК МГУ}\\
	{Галустова Артемия Львовича}
    \end{center}
    
    \begin{center}
	\vfill
	{\small гор. Москва\\2020 год}
    \end{center}
\end{titlepage}

\tableofcontents
\newpage

\section{Подвариант 1}
\subsection{Цель работы}

Изучить классический метод Гаусса (а также модифицированный метод Гаусса),
применяемый для решения системы линейных алгебраических уравнений.

\subsection{Постановка задачи}

Дана система уравнений $Ax=f$ порядка $n \times n$ с невырожденной матрицей $A$. Написать
программу, решающую систему линейных алгебраических уравнений заданного
пользователем размера ($n$ – параметр программы) методом Гаусса и методом Гаусса с
выбором главного элемента.

\subsection{Цели и задачи практической работы}

\begin{enumerate}
\item
Решить заданную СЛАУ методом Гаусса и методом Гаусса с выбором главного
элемента;
\item
Вычислить определитель матрицы $det(A)$;
\item
Вычислить обратную матрицу $A^{-1}$;
\item
Определить число обусловленности $M_A=||A|| \times ||A^{-1}||$;
\item
Исследовать вопрос вычислительной устойчивости метода Гаусса (при больших
значениях параметра $n$);
\item
Правильность решения СЛАУ подтвердить системой тестов (использован онлайн сервис символьных вычислений Wolfaram One).
\end{enumerate}

\newpage
\subsection{Описание метода решения}
\subsubsection{Метод Гаусса}
Рассмотрим СЛАУ вида
\begin{align*}
\begin{cases}
a_{11} x_1 + a_{12} x_2 + ... + a_{1n} x_n = f_1\\
a_{21} x_1 + a_{22} x_2 + ... + a_{2n} x_n = f_2\\
...\\
a_{n1} x_1 + a_{n2} x_2 + ... + a_{nn} x_n = f_n\\
\end{cases}
\end{align*}
\par
Которая соответствует матричному уравнению $A \times x = f$, где $A = (a_{ij}), f = (f_1, f_2, ... f_n)^T$ и $|A| \neq 0$. Поиск решения методом Гаусса, как с выбором главного элемента, так и без, состоит из двух основных этапов: прямой ход, приведение матрицы к треугольному виду виду; обратный ход, последовательное нахождение $x_n, x_{n - 1}, ... x_1$. Опишем подробнее данные этапы.
\par
Прямой ход метода Гаусса осуществляется следующим образом. В первой строке матрицы системы выбирается либо первый ненулевой, в случае обычного метода Гаусса, либо наибольший по модулю, в случае метода с выбором главного элемента, элемент. Такой ненулевой элемент обязательно найдётся в силу невырожденности матрицы. Затем столбцы расширенной матрицы переставляются так, чтобы выбранный элемент был в первом столбце. Это возможно реализовать явной перестановкой столбцов матрицы и перенумерованием переменных, однако это неэффективно, т.к. требует обмена значений $n$ чисел на каждом шаге. В представленной реализации сохраняется дополнительный массив, указывающий порядок выбора столбцов на каждом шаге прямого хода. Не ограничивая общности, можно считать, что такой элемент находится в первом столбце. Затем из всех последующих строк расширенной матрицы вычитается первая, домноженная на коэффициент $\frac{a_{1k}}{a_{11}}, k = 2 .. n$. Получается система
\begin{equation*}
\left\{
\begin{alignedat}{3}
a_{11} x_1 + &a_{12} x_2 &+ ... + a_{1n} x_n & = f_1\\
           &a'_{22} x_2 &+ ... + a'_{2n} x_n & = f'_2\\
&...&&\\
           &a'_{n2} x_2 &+ ... + a'_{nn} x_n & = f'_n\\
\end{alignedat}
\right.
\end{equation*}
\\
где $a'_{ij} = a_{ij} - a_{i1} \frac{a_{1j}}{a_{11}}, f'_i = f_i - a_{i1} \frac{f_1}{a_{11}}$. Далее указанные преобразования аналогично повторяются для второй, третей и т.д. строк. В итоге по результатам прямого хода получается система с верхнетреугольной матрицей. Т.к. осуществлялись только сложения строк с коэффициентом, то данная система эквивалентна изначальной. Представление в виде системы уравнений:
\begin{equation*}
\left\{
\begin{alignedat}{3}
&c_{11} x_1 + & c_{12} x_2 + ... +&c_{1 n-1} x_{n-1} +     &c_{1n} x_n    & = g_1\\
  &           & c_{22} x_2 + ... +&c_{2 n-1} x_{n-1} +     &c_{2n} x_n    & = g_2\\
&&...&&&\\
&             &                   &c_{n - 1 n-1} x_{n-1} + &c_{n-1 n} x_n & = g_{n-1}\\
 &            &                   &                        &c_{nn} x_n    & = g_n\\
\end{alignedat}
\right.
\end{equation*}
\par
Обратный ход метода Гаусса осуществляется над полученной при прямом ходе системой с учётном перестановки столбцов. Корни уравнения находятся в обратном порядке с использованием найденных ранее по формуле $x_k = c^{-1}_{kk} (g_k - c_{k k+1}x_{k+1} - c_{k k+2}x_{k+2} + ... + - c_{k n}x_n)$. Заметим, что $c_{kk} \neq 0$ т.к. выбирается ненулевым во время прямого хода.

\subsubsection{Вычисление определителя матрицы}
Воспользовавшись прямым ходом метода Гаусса, описанным выше, можно получить верхнюю треугольную матрицу с точностью до перестановки столбцов. Тогда для вычисления определителя достаточно перемножить элементы соответствующие диагональным и домножить на $(-1)^{N(p)}$, где $N(p)$ -- число инверсий в перестановке порядка столбцов.
\subsubsection{Вычисление обратной матрицы}
Обратную матрицу вычислим по методу Гаусса-Жордана. В приведенном выше методе Гаусса (без выбора главного элемента) сделаем изменение. В случае, если диагональный элемент равен $0$ будем обменивать строки а не столбцы и обменивать в явном виде (а не сохранять порядок). Также на каждом шаге будем нормализовать текущую строку, т.е. делить строку на диагональный элемент. Тогда применение к расширенной матрице $(A|I)$ метод Гаусса с указанными модификациями дважды,  в прямом и "обратном" направлении, получим матрицу $(I|A^{-1})$, правая часть этой расширенной матрицы и содержит обратную матрицу.
\subsubsection{Вычисление числа обусловленности}
Для вычисления числа обусловленности воспользуемся нормой $||A||_{\infty } = \max\limits_{1 \leq i \leq n} \sum\limits_{j=1}^n |a_{ij}|$. Тогда число обусловленности СЛАУ $M_A=||A|| \cdot ||A^{-1}||$, где обратную матрицу можно вычислить приведённым выше методом.
\subsection{Описание программы}
Данный алгоритм был реализован на языке С с применением стандартной библиотеки языка С. Ниже приведена часть файла algs.c содержащая реализацию описанных выше методов. Полный код программы доступен в разделе <<\nameref{source}>>, а также онлайн по адресу \url{https://github.com/NotLebedev/le_solver}.
\lstinputlisting[style=CStyle, firstline=16]{../algs.c}
\subsection{Тестирование}
Для проверки результатов выполнения программы на тестах был использован использован онлайн сервис символьных вычислений Wolfaram One. Для проверки решений уравнений использовалась системная функция {\ttfamily LinearSolve}, {\ttfamily Det} для вычисления определителя, {\ttfamily Inverse} для вычисления обратной матрицы. Для Нахождения числа обусловленности была использована функция {\ttfamily  cond[a\_] := Norm[a, Infinity]*Norm[Inverse[a], Infinity]}. Погрешность вычислений проверялась при помощи функции {\ttfamily EuclidianDistance}, т.е. находилось расстояние между векторами решений и обратными матрицами, найденными программой и Wolfram One.
\begin{enumerate}
\item
Матрица системы:
\begin{align*}
A = \begin{pmatrix}
2&  2&  -1&  1 \\
4&  3&  -1&  2 \\
8&  5&  -3&  4 \\
3&  3&  -2&  4 \\
\end{pmatrix}
\end{align*}
Столбец свободных членов:
\begin{align*}
f = \begin{pmatrix}
4 \\
6 \\
12 \\
6 \\
\end{pmatrix}
\end{align*}
Решение найденное методом Гаусса:
\begin{align*}
x = \begin{pmatrix}
0,60000 \\
   1,00000 \\
  -1,00000 \\
  -0,20000 \\
\end{pmatrix}
\end{align*}
Решение найденное методом Гаусса с выбором главного элемента:
\begin{align*}
x = \begin{pmatrix}
0,60000 \\
   1,00000 \\
  -1,00000 \\
  -0,20000 \\
\end{pmatrix}
\end{align*}
Обратная матрица
\begin{align*}
A^{-1} = \begin{pmatrix}
-0,60000&     0,10000&     0,30000&    -0,20000 \\
   1,00000&     0,50000&    -0,50000&     0,00000 \\
  -1,00000&     1,50000&    -0,50000&     0,00000 \\
  -0,80000&     0,30000&    -0,10000&     0,40000 \\
\end{pmatrix}
\end{align*}
Определитель вычисленный без выбора главного элемента  $\Delta = 10,000000$.\\
Определитель вычисленный с выбором главного элемента $\Delta = 10,000000$.\\
Число обусловленности $M_A = 60,000000$.\\
Погрешность всех величин не превосходит $3 \times 10^{-16}$.
\item
Матрица системы:
\begin{align*}
A = \begin{pmatrix}
1&     1&     3&    -2 \\
   2&     2&     4&    -1 \\
   3&     3&     5&    -2 \\
   2&     2&     8&    -3 \\
\end{pmatrix}
\end{align*}
Столбец свободных членов:
\begin{align*}
f = \begin{pmatrix}
1 \\
2 \\
1 \\
2 \\
\end{pmatrix}
\end{align*}
Определитель вычисленный без выбора главного элемента  $\Delta = 0$.\\
Определитель вычисленный с выбором главного элемента $\Delta = 0$.\\
Матрица системы вырождена, это подтверждается Wolfram One.
\item
Матрица системы:
\begin{align*}
A = \begin{pmatrix}
2&           5&          -8&           3 \\
         4&           3&          -9&           1 \\
         2&           3&          -5&          -6 \\
         1&           8&          -7&           0 \\
\end{pmatrix}
\end{align*}
Столбец свободных членов:
\begin{align*}
f = \begin{pmatrix}
8 \\
         9 \\
         7 \\
        12 \\
\end{pmatrix}
\end{align*}
Решение найденное методом Гаусса:
\begin{align*}
x = \begin{pmatrix}
3,00000 \\
   2,00000 \\
   1,00000 \\
   0,00000 \\
\end{pmatrix}
\end{align*}
Решение найденное методом Гаусса с выбором главного элемента:
\begin{align*}
x = \begin{pmatrix}
3,00000 \\
   2,00000 \\
   1,00000 \\
   0,00000 \\
\end{pmatrix}
\end{align*}
Обратная матрица
\begin{align*}
A^{-1} = \begin{pmatrix}
  -1,71958&     1,22222&    -0,65608&     0,86243 \\
  -0,65079&     0,33333&    -0,26984&     0,50794 \\
  -0,98942&     0,55556&    -0,40212&     0,56085 \\
  -0,07407&     0,11111&    -0,18519&     0,07407 \\
\end{pmatrix}
\end{align*}
Определитель вычисленный без выбора главного элемента  $\Delta = -189,000000$.\\
Определитель вычисленный с выбором главного элемента $\Delta = -189,000000$.\\
Число обусловленности $M_A = 80,285714$.\\
Погрешность решений и определителей не превосходит $2 \times 10^{-17}$.\\
Погрешность обратной матрицы и числа обусловленности не превосходит $10^{-6}$.

\item
Матрица системы:
\begin{align*}
A = \begin{pmatrix}
0&     0&     0&    65&     0 \\
   4&     9&     1&     7&    10 \\
   3&    27&     8&     1&     1 \\
   7&    32&     1&    61&     0 \\
   4&    69&     2&     0&    45 \\
\end{pmatrix}
\end{align*}
Столбец свободных членов:
\begin{align*}
f = \begin{pmatrix}
   0 \\
  65 \\
  23 \\
  11 \\
   3 \\
\end{pmatrix}
\end{align*}
Решение найденное методом Гаусса:
\begin{align*}
x = \begin{pmatrix}
13,61231 \\
  -2,85385 \\
   7,03716 \\
   0,00000 \\
   2,91983 \\
\end{pmatrix}
\end{align*}
Решение найденное методом Гаусса с выбором главного элемента:
\begin{align*}
x = \begin{pmatrix}
  13,61231 \\
  -2,85385 \\
   7,03716 \\
   0,00000 \\
   2,91983 \\
\end{pmatrix}
\end{align*}
Обратная матрица:
\begin{align*}
A^{-1} = \begin{pmatrix}
  -0,07755&     0,20936&    -0,02203&     0,05897&    -0,04604 \\
  -0,01466&    -0,04823&     0,00071&     0,02114&     0,01070 \\
   0,07336&     0,07777&     0,13149&    -0,08925&    -0,02021 \\
   0,01538&     0,00000&     0,00000&     0,00000&     0,00000 \\
   0,02610&     0,05188&    -0,00498&    -0,03369&     0,01080 \\
\end{pmatrix}
\end{align*}
Определитель вычисленный без выбора главного элемента  $\Delta = 3200925,000000$.\\
Определитель вычисленный с выбором главного элемента $\Delta = 3200925,000000$.\\
Число обусловленности $M_A = 49,673854$.\\
Погрешность решений и определителей не превосходит $2 \times 10^{-17}$.\\
Погрешность обратной матрицы и числа обусловленности не превосходит $10^{-6}$.

\item
Матрица размером $n \times n, n = 30$ задаётся по формуле
\begin{align}
A_{ij} = \left\{
\begin{array}{ll}
\frac{i+j}{m+n}, i \neq j\\
n + m^2 + \frac{j}{m} + \frac{i}{m}, i = j
\end{array}
\right.
\end{align}
где $1 \leq i,j \leq n; m = 20$. А столбец свободных членов по формуле $f_i = mi + n$. Приведём также часть решения полученного программой и числовые характеристики.\\
Решение найденное методом Гаусса:
\begin{align*}
x = \begin{pmatrix}
0,05067 \\
   0,09618 \\
   0,14169 \\
   ...\\
   1,27629 \\
   1,32155 \\
   1,36681 \\
\end{pmatrix}
\end{align*}
Решение найденное методом Гаусса с выбором главного элемента:
\begin{align*}
x = \begin{pmatrix}
  0,05067 \\
   0,09618 \\
   0,14169 \\
   ...\\
   1,27629 \\
   1,32155 \\
   1,36681 \\
\end{pmatrix}
\end{align*}
Обратная матрица:
\begin{align*}
A^{-1} = \begin{pmatrix}
    0,00233&    -0,00000&    -0,00000& ... &0,00000\\
   -0,00000&     0,00233&    -0,00000& ... &0,00000\\
   -0,00000&    -0,00000&     0,00232& ... &0,00000\\
   ...\\
   -0,00000&    -0,00000&    -0,00000& ... &0,00231\\
\end{pmatrix}
\end{align*}
Определитель матрицы этой системы при любом вычислении получается большим $\Delta = 1,09699 \times 10^{79}$.\\
Число обусловленности $M_A = 1,116706$.\\
Точность вычислений для этой матрицы существенно хуже чем для предыдущих, порядка $10^{-6}$ для всех величин, кроме определителя, его погрешность $1,61243 \times 10^{67}$, однако, относительная погрешность тоже порядка $10^{-6}$.
\item
Матрица размером $n \times n, n = 100$ задаётся по формуле
\begin{align}
A_{ij} = \left\{
\begin{array}{ll}
\frac{i+j}{m+n}, i \neq j\\
\frac{n + m^2 + \frac{j}{m} + \frac{i}{m}}{400}, i = j
\end{array}
\right.
\end{align}
где $1 \leq i,j \leq n; m = 30$. А столбец свободных членов по формуле $f_i = mi + n$. Приведём также часть решения полученного программой и числовые характеристики.\\
Решение найденное методом Гаусса:
\begin{align*}
x = \begin{pmatrix}
171,94735 \\
 170,51866 \\
 169,07228 \\
   ...\\
-166,23950 \\
-174,93024 \\
-183,88983 \\
\end{pmatrix}
\end{align*}
Решение найденное методом Гаусса с выбором главного элемента:
\begin{align*}
x = \begin{pmatrix}
171,94735 \\
 170,51866 \\
 169,07228 \\
   ...\\
-166,23950 \\
-174,93024 \\
-183,88983 \\
\end{pmatrix}
\end{align*}
Обратная матрица:
\begin{align*}
A^{-1} = \begin{pmatrix}
    0,38147&    -0,01837&    -0,01822& ... &     0,02038 \\
	0,01837&     0,38424&    -0,01806& ... &     0,02004 \\
	0,01822&    -0,01806&     0,38704& ... &     0,01971 \\
	...\\
	0,02038&     0,02004&     0,01971& ... &     0,95021 \\
\end{pmatrix}
\end{align*}
Определитель матрицы этой системы при любом вычислении получается большим $\Delta = -6,06825 \times 10^{24}$.\\
Число обусловленности $M_A = 320,755833$.\\
Из-за плохой обусловленности матрицы системы возросли и погрешности. Погрешности порядка $10^{-3}$, кроме определителя, абсолютная погрешность составляет $2 \times 10^{17}$, относительная порядка $5 \times 10^{-8}$.
\end{enumerate}
\subsection{Выводы}
При выполнении поставленных целей был подробно разобран и реализован метод Гаусса. Даже с некоторыми оптимизациями метод прост для реализации и отладки. Вычисление решений для систем небольшого порядка очень удобно, т.к. метод работает за фиксированное время и обладает достаточной для таких случаев точностью. При необходимости решать плохо обусловленные системы больших размеров данный метод выдаёт гораздо меньшую точность, проявляя вычислительную неустойчивость.

\newpage
\section{Подвариант 2}
\subsection{Цель работы}
Изучить классические итерационные методы (Зейделя и верхней релаксации),
используемые для численного решения систем линейных алгебраических уравнений;
изучить скорость сходимости этих методов в зависимости от выбора итерационного
параметра.
\subsection{Постановка задачи}
Дана система уравнений $A \times x=f$ порядка $n \times n$ с невырожденной матрицей A. Написать
программу численного решения данной системы линейных алгебраических уравнений
($n$ – параметр программы), использующую численный алгоритм итерационного метода
верхней релаксации:
\begin{align*}
(D + \omega A^{(-)})\frac{x^{k+1}-x^k}{\omega} + Ax^k = f
\end{align*}
где $D, A^{(-)}$ - диагональная и нижняя треугольная матрица соответственно, а $\omega$ - итерационный параметр (при $\omega = 1$ метод верхней релаксации совпадает с методом Зейделя, $0 < \omega < 2$).
\par
Предусмотреть возможность задания элементов матрицы системы и ее правой части как
во входном файле данных, так и путем задания специальных формул.
\subsection{Цели и задачи практической работы}
\begin{enumerate}
\item
Решить заданную СЛАУ итерационным методом Зейделя (или более общим
методом верхней релаксации);
\item
Разработать критерий остановки итерационного процесса, гарантирующий
получение приближенного решения исходной системы СЛАУ с заданной
точностью;
\item
Изучить скорость сходимости итераций к точному решению задачи (при
использовании итерационного метода верхней релаксации провести эксперименты
с различными значениями итерационного параметра $\omega$ (в случае симметрической
положительно определенной матрицы системы известно, что для сходимости
итераций следует выбирать $0 < \omega < 2$ ; при $\omega = 1$ метод верхней релаксации
совпадает с методом Зейделя);
\item
Правильность решения СЛАУ подтвердить системой тестов (использован онлайн сервис символьных вычислений Wolfaram One).
\end{enumerate}

\newpage
\subsection{Описание метода решения}
Рассмотрим СЛАУ вида
\begin{align*}
\begin{cases}
b_{11} x_1 + b_{12} x_2 + ... + b_{1n} x_n = g_1\\
b_{21} x_1 + b_{22} x_2 + ... + b_{2n} x_n = g_2\\
...\\
b_{n1} x_1 + b_{n2} x_2 + ... + b_{nn} x_n = g_n\\
\end{cases}
\end{align*}
\par
Которая соответствует матричному уравнению $B \times x = g$, где $B = (a_{ij}), g = (g_1, g_2, ... g_n)^T$ и $|B| \neq 0$. Для сходимости метода верхней релаксации необходимо, чтобы матрица системы была самосопряженной. Для этого домножим уравнение с обоих сторон на $B^T$, получим $B^T B \times x = B^T g$, переобозначив $A = B^T B, f = B^T g$ получим уравнение $A \times x = f$, которое эквивалентно изначальному и с которым будет осуществлять работу алгоритм.
\par
В покомпонентном виде итерационная формула имеет вид
\begin{align*}
x^{k+1}_i = x^k_i + \frac{\omega}{a_{ii}}\left(f_i - \sum\limits^{i-1}_{j=1}a_{ij}x^{k+1}_j - \sum\limits^{n}_{j=i}a_{ij}x^{k}_j\right), i = 1,...,n
\end{align*}
Заметим, что в этой формуле для $i$ элемента используются ранее вычисленные элементы вектора решения c $1$ до $i-1$ и элементы вектора решения с предыдущей итерации, но уже с $i$ по $n$. Таким образом нет необходимости хранить два вектора решения и формулу можно переписать в виде
\begin{align*}
x_i := x_i + \frac{\omega}{a_{ii}}\left(f_i - \sum\limits^{n}_{j=1}a_{ij}x_j\right), i = 1,...,n
\end{align*}
где вектор $x$ содержит в начале каждой итерации решение с предыдущей итерации, а в конце новое приближение решения. Таким образом алгоритм значительно упрощается и уменьшает потребление дополнительной памяти.
\par
Критерием остановки процесса является $||x^{k+1} - x^{k}|| < \varepsilon$, используется Евклидова норма. Получим эту оценку. Запишем разность двух итерационных приближений.
\begin{align*}
|x^{k+1}_i - x^k_i| = \frac{\omega}{|a_{ii}|} | f_i - \sum\limits^{n}_{j=1}a_{ij}x^k_j|, i = 1,...,n
\end{align*}
Если итерационный метод сходится, то невязка $\Psi_k = ||f_i - \sum\limits^{n}_{j=1}a_{ij}x^k_j||$ убывает c ростом номера $k$. Из этих двух равенств следует, что
\begin{align*}
||x^{k+1} - x^{k}|| \leq \frac{\omega \Psi_k}{m_a}\\
m_a = \max\limits_{1 \leq i \leq n} |a_{ii}|
\end{align*}

И в силу убывания невязки критерий остановки корректен.
\subsection{Описание программы}
Данный алгоритм был реализован на языке С с применением стандартной библиотеки языка С. Ниже приведена часть файла relaxation.c содержащая реализацию описанного выше метода. Полный код программы доступен в разделе <<\nameref{source}>>, а также онлайн по адресу \url{https://github.com/NotLebedev/le_solver}.
\lstinputlisting[style=CStyle, firstline=6]{../relaxation.c}

\subsection{Тестирование}
Для проверки результатов выполнения программы на тестах был использован использован онлайн сервис символьных вычислений Wolfaram One. Для проверки решений уравнений использовалась системная функция {\ttfamily LinearSolve}. Погрешность вычислений проверялась при помощи функции {\ttfamily EuclidianDistance}, т.е. находилось расстояние между векторами решений, найденными программой и Wolfram One. Т.к. для тестирования использовались те же самые данные, что и в первом подварианте, то взято вычисленное ранее число обусловленности для иллюстрации работы алгоритма.

\begin{enumerate}
\item
Матрица системы:
\begin{align*}
A = \begin{pmatrix}
2&  2&  -1&  1 \\
4&  3&  -1&  2 \\
8&  5&  -3&  4 \\
3&  3&  -2&  4 \\
\end{pmatrix}
\end{align*}
Столбец свободных членов:
\begin{align*}
f = \begin{pmatrix}
4 \\
6 \\
12 \\
6 \\
\end{pmatrix}
\end{align*}
Решение найденное методом верхней релаксации:
\begin{align*}
x = \begin{pmatrix}
0,60000 \\
   1,00000 \\
  -1,00000 \\
  -0,20000 \\
\end{pmatrix}
\end{align*}
Количество итераций в зависимости от параметра $\omega$:
\begin{center}
\begin{tabular}{|c|c|}
\hline
$\omega$ & число итераций\\
\hline
0,1 & 18543\\
0,3 & 6009\\
0,5 & 3277\\
0,7 & 1995\\
 1 & 1054\\
 1,3 & 985\\
 1,5 & 1260\\
 1,7 & 2105\\
 \hline
\end{tabular}
\end{center}

Число обусловленности $M_A = 60,000000$.\\
Решение в точности совпадает с вычисленным в Wolfram One.
\item
Матрица системы:
\begin{align*}
A = \begin{pmatrix}
1&     1&     3&    -2 \\
   2&     2&     4&    -1 \\
   3&     3&     5&    -2 \\
   2&     2&     8&    -3 \\
\end{pmatrix}
\end{align*}
Столбец свободных членов:
\begin{align*}
f = \begin{pmatrix}
1 \\
2 \\
1 \\
2 \\
\end{pmatrix}
\end{align*}
Матрица системы вырождена, как было показано ранее. Однако, применение алгоритма с $\omega = 1$ даёт результат:
\begin{align*}
x = \begin{pmatrix}
0,05556 \\
  -0,00000\\ 
   0,38889 \\
   0,33333 \\
\end{pmatrix}
\end{align*}
Это демонстрирует, что применение алгоритма к системам с вырожденной матрицей даёт некорректный результат и никак не указывает на произошедшую ошибку.

\item
Матрица системы:
\begin{align*}
A = \begin{pmatrix}
2&           5&          -8&           3 \\
         4&           3&          -9&           1 \\
         2&           3&          -5&          -6 \\
         1&           8&          -7&           0 \\
\end{pmatrix}
\end{align*}
Столбец свободных членов:
\begin{align*}
f = \begin{pmatrix}
8 \\
         9 \\
         7 \\
        12 \\
\end{pmatrix}
\end{align*}
Решение найденное методом верхней релаксации:
\begin{align*}
x = \begin{pmatrix}
3,00000 \\
   2,00000 \\
   1,00000 \\
  -0,00000 \\
\end{pmatrix}
\end{align*}
Количество итераций в зависимости от параметра $\omega$:
\begin{center}
\begin{tabular}{|c|c|}
\hline
$\omega$ & число итераций\\
\hline
0,1 & 90579\\
0,3 & 28981\\
0,5 & 15819\\
0,7 & 9960\\
 1 & 5363\\
 1,3 & 2876\\
 1,5 & 1972\\
 1,7 & 1085\\
 \hline
\end{tabular}
\end{center}

Число обусловленности $M_A = 80,285714$.\\
Погрешность решения не превосходит $2 \times 10^{-17}$.

\item
Матрица системы:
\begin{align*}
A = \begin{pmatrix}
0&     0&     0&    65&     0 \\
   4&     9&     1&     7&    10 \\
   3&    27&     8&     1&     1 \\
   7&    32&     1&    61&     0 \\
   4&    69&     2&     0&    45 \\
\end{pmatrix}
\end{align*}
Столбец свободных членов:
\begin{align*}
f = \begin{pmatrix}
   0 \\
  65 \\
  23 \\
  11 \\
   3 \\
\end{pmatrix}
\end{align*}
Решение найденное методом верхней релаксации:
\begin{align*}
x = \begin{pmatrix}
13,61230 \\
  -2,85385 \\
   7,03717 \\
   0,00000 \\
   2,91983 \\
\end{pmatrix}
\end{align*}
Количество итераций в зависимости от параметра $\omega$:
\begin{center}
\begin{tabular}{|c|c|}
\hline
$\omega$ & число итераций\\
\hline
0,1 & 6024\\
0,3 & 1905\\
0,5 & 1035\\
0,7 & 648\\
 1 & 343\\
 1,3 & 140\\
 1,5 & 144\\
 1,7 & 241\\
 \hline
\end{tabular}
\end{center}

Число обусловленности $M_A = 49,673854$.\\
Погрешность решения не превосходит $3 \times 10^{-6}$.

\item
Матрица размером $n \times n, n = 30$ задаётся по формуле
\begin{align}
A_{ij} = \left\{
\begin{array}{ll}
\frac{i+j}{m+n}, i \neq j\\
n + m^2 + \frac{j}{m} + \frac{i}{m}, i = j
\end{array}
\right.
\end{align}
где $1 \leq i,j \leq n; m = 20$. А столбец свободных членов по формуле $f_i = mi + n$. Приведём также часть решения полученного методом верхней релаксации:
\begin{align*}
x = \begin{pmatrix}
0,05067 \\
   0,09618 \\
   0,14169 \\
   0,18718 \\
   ...\\
   1,27629 \\
   1,32155 \\
   1,36681 \\
\end{pmatrix}
\end{align*}
Количество итераций в зависимости от параметра $\omega$:
\begin{center}
\begin{tabular}{|c|c|}
\hline
$\omega$ & число итераций\\
\hline
0,1 & 203\\
0,3 & 64\\
0,5 & 35\\
0,7 & 21\\
 1 & 8\\
 1,3 & 23\\
 1,5 & 39\\
 1,7 & 74\\
 \hline
\end{tabular}
\end{center}

Число обусловленности $M_A = 1,116706$.\\
Погрешность решения не превосходит $10^{-9}$.

\item
Матрица размером $n \times n, n = 100$ задаётся по формуле
\begin{align}
A_{ij} = \left\{
\begin{array}{ll}
\frac{i+j}{m+n}, i \neq j\\
\frac{n + m^2 + \frac{j}{m} + \frac{i}{m}}{400}, i = j
\end{array}
\right.
\end{align}
где $1 \leq i,j \leq n; m = 30$. А столбец свободных членов по формуле $f_i = mi + n$. Приведём также часть решения полученного методом верхней релаксации:
\begin{align*}
x = \begin{pmatrix}
171,94735 \\
 170,51866 \\
 169,07228 \\
   ...\\
 -166,23950 \\
-174,93024 \\
-183,88983 \\  
\end{pmatrix}
\end{align*}
Количество итераций в зависимости от параметра $\omega$:
\begin{center}
\begin{tabular}{|c|c|}
\hline
$\omega$ & число итераций\\
\hline
0,01 & 245547\\
0,05 & 55154\\
0,1 & 28525\\
0,3 & 28908\\
0,5 & 51326\\
0,7 & 81598\\
 1 & 150577\\
 1,3 & 278681\\
 1,5 & 450714\\
 1,7 & 852021\\
 \hline
\end{tabular}
\end{center}

Число обусловленности $M_A = 320,755833$.\\
Погрешность решения не превосходит $2 \times 10^{-7}$.
\end{enumerate}

\subsection{Выводы}
При выполнении поставленных целей был подробно разобран и реализован метод верхней релаксации. Он ещё проще в реализации, чем метод Гаусса. Данный метод подходит для хорошо обусловленных систем и систем с диагональным преобладанием, однако для плохо обусловленных систем скорость работы алгоритма неудовлетворительна. Также метод требует дополнительных вычислений для решения систем, про которые не известно, что они имеют диагональное преобладание. Особенностью метода является возможность выбора итерационного параметра, удачный выбор которого может уменьшить количество итераций более чем в $80$ раз, но оптимальное значение параметра зависит от особенностей системы.

\newpage
\section{Полный листинг программы} \label{source}

\subsection{Файл main.c}
\lstinputlisting[style = CStyle]{../main.c}
\subsection{Файл algs.h}
\lstinputlisting[style = CStyle]{../algs.h}
\subsection{Файл algs.c}
\lstinputlisting[style = CStyle]{../algs.c}
\subsection{Файл matrix.h}
\lstinputlisting[style = CStyle]{../matrix.h}
\subsection{Файл matrix.c}
\lstinputlisting[style = CStyle]{../matrix.c}
\subsection{Файл relaxation.h}
\lstinputlisting[style = CStyle]{../relaxation.h}
\subsection{Файл relaxation.c}
\lstinputlisting[style = CStyle]{../relaxation.c}
\subsection{Файл error.h}
\lstinputlisting[style = CStyle]{../error.h}
\end{document}