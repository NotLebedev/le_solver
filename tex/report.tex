\documentclass[a4paper,12pt,titlepage,finall]{article}

\usepackage[T1,T2A]{fontenc}     % форматы шрифтов
\usepackage[utf8]{inputenc}      % кодировка символов, используемая в данном файле
\usepackage[russian]{babel}      % пакет русификации
\usepackage{tikz}                % для создания иллюстраций
\usepackage{pgfplots}            % для вывода графиков функций
\usepackage{geometry}		     % для настройки размера полей
\usepackage{indentfirst}         % для отступа в первом абзаце секции
\usepackage{amsmath,amsthm,amssymb}
\usepackage{mathtext}
\usepackage{graphicx}
\graphicspath{ {./} }

%Настройка листингов для языка C
\usepackage{xcolor}
\usepackage{listings}

\definecolor{mGreen}{rgb}{0,0.6,0}
\definecolor{mGray}{rgb}{0.5,0.5,0.5}
\definecolor{mPurple}{rgb}{0.58,0,0.82}
\definecolor{backgroundColour}{rgb}{0.95,0.95,0.92}

\lstdefinestyle{CStyle}{
    backgroundcolor=\color{backgroundColour},   
    commentstyle=\color{mGreen},
    keywordstyle=\color{magenta},
    numberstyle=\tiny\color{mGray},
    stringstyle=\color{mPurple},
    basicstyle=\footnotesize,
    breakatwhitespace=false,         
    breaklines=true,                 
    captionpos=b,                    
    keepspaces=true,                 
    numbers=none,                    
    numbersep=5pt,                  
    showspaces=false,                
    showstringspaces=false,
    showtabs=false,                  
    tabsize=2,
    language=C,
}

\lstdefinestyle{MakeStyle}{
    backgroundcolor=\color{backgroundColour},   
    commentstyle=\color{mGreen},
    keywordstyle=\color{magenta},
    numberstyle=\tiny\color{mGray},
    stringstyle=\color{mPurple},
    basicstyle=\footnotesize,
    breakatwhitespace=false,         
    breaklines=true,                 
    captionpos=b,                    
    keepspaces=true,                 
    numbers=none,                    
    numbersep=5pt,                  
    showspaces=false,                
    showstringspaces=false,
    showtabs=false,                  
    tabsize=2,
    language=[gnu] make,
}

% выбираем размер листа А4, все поля ставим по 3см
\geometry{a4paper,left=30mm,top=30mm,bottom=30mm,right=30mm}

\setcounter{secnumdepth}{0}      % отключаем нумерацию секций
%\setcounter{tocdepth}{1}

\usepgfplotslibrary{fillbetween} % для изображения областей на графиках

\begin{document}
\begin{titlepage}
    \begin{center}
	{\small \sc Московский государственный университет \\имени М.~В.~Ломоносова\\
	Факультет вычислительной математики и кибернетики\\}
	\hrulefill
	\vfill
	{\large \bf Компьютерный практикум по учебному курсу}\\
	~\\
	{\Large \bf <<ВВЕДЕНИЕ В ЧИСЛЕННЫЕ МЕТОДЫ>>}\\ 
	~\\
	{\Large \bf ЗАДАНИЕ № 1}\\
	~\\
	{\large \bf ОТЧЕТ}\\
	{\bf о выполненном задании}\\
	{студента 201 учебной группы факультета ВМК МГУ}\\
	{Галустова Артемия Львовича}
    \end{center}
    
    \begin{center}
	\vfill
	{\small гор. Москва\\2020 год}
    \end{center}
\end{titlepage}

\tableofcontents
\newpage

\section{Подвариант 1}
\subsection{Цель работы}

Изучить классический метод Гаусса (а также модифицированный метод Гаусса),
применяемый для решения системы линейных алгебраических уравнений.

\subsection{Постановка задачи}

Дана система уравнений $Ax=f$ порядка $n \times n$ с невырожденной матрицей $A$. Написать
программу, решающую систему линейных алгебраических уравнений заданного
пользователем размера ($n$ – параметр программы) методом Гаусса и методом Гаусса с
выбором главного элемента.

\subsection{Цели и задачи практической работы}

\begin{enumerate}
\item
Решить заданную СЛАУ методом Гаусса и методом Гаусса с выбором главного
элемента;
\item
Вычислить определитель матрицы $det(A)$;
\item
Вычислить обратную матрицу $A^{-1}$;
\item
Определить число обусловленности $МA=||A|| \times ||A^{-1}||$;
\item
Исследовать вопрос вычислительной устойчивости метода Гаусса (при больших
значениях параметра $n$);
\item
Правильность решения СЛАУ подтвердить системой тестов (использован пакет символьных вычислений Wolfaram Mathematica).
\end{enumerate}
\end{document}